%% LyX 2.0.6 created this file.  For more info, see http://www.lyx.org/.
%% Do not edit unless you really know what you are doing.
\documentclass[english]{article}
\usepackage[T1]{fontenc}
\usepackage[latin9]{inputenc}
\usepackage{graphicx}

\makeatletter

%%%%%%%%%%%%%%%%%%%%%%%%%%%%%% LyX specific LaTeX commands.
%% Because html converters don't know tabularnewline
\providecommand{\tabularnewline}{\\}
%% A simple dot to overcome graphicx limitations
\newcommand{\lyxdot}{.}


%%%%%%%%%%%%%%%%%%%%%%%%%%%%%% User specified LaTeX commands.
\usepackage{babel}

\makeatother

\usepackage{babel}
\begin{document}

\section{Trigger processor ADDC Interface}

The ART data from an entire wedge will be transmitted to a single
trigger processor. 32 ADDCs are required to cover one wedge resulting
in 32 total fibers per Trigger Processor. Since the MM and sTGC trigger
processors will share the same ATCA carrier card, each carrier will
support 2 wedges. 

The ART Data from the ADDC will be transmitted using the GigaBit Transceiver
(GBT) architecture and transmission protocol in a low-latency widebus
mode. The Trigger Processor will take advantage of the GBT firmware
developed by GBT Project to implement the receivers.


\subsection{ART Data Protocol}

The GBT packet in widebus mode will provide 112 data bits and arrives
once every bunch crossing. One ADDC will service 32 VMMs and each
packet can contain ART data from a maximum of 8 triggered VMMs. Each
VMM will be uniquely identifed to determine which micromegas strip
on the wedge was hit.

There are two options for how data packet bits will be defined. The
difference between the two is how the VMM ID information is encoded.
The first data protocol option will provide the VMM IDs of every VMM
that was triggered by asserting a bit in a 32-bit hit list. The second
option will encode each VMM ID in a list. For both options the triggered
strip number within each VMM will be provided in a list. The first
option would move the VMM ID encoding task from the ADDC ASIC to the
Trigger Processor FPGA.

\begin{tabular}{|c|c|c|c||c|c|}
\hline 
\multicolumn{6}{|c|}{Option 1 GBT DATA{[}111:56{]}}\tabularnewline
\hline 
\hline 
``1010''  & BCID{[}11:0{]}  & ERR\_FLAGS{[}7:0{]}  & HIT\_LIST{[}31:0{]}  & ARTDATA\_PARITY{[}7:0{]}  & 8 x ARTDATAx{[}5:0{]}\tabularnewline
\hline 
\end{tabular}

\begin{tabular}{|c|c|c||c|c||c|}
\hline 
\multicolumn{6}{|c|}{Option 2 GBT DATA{[}111:56{]}}\tabularnewline
\hline 
\hline 
HIT\_CNT{[}3:0{]}  & BCID{[}11:0{]}  & \multicolumn{2}{c|}{8 x VMMIDx{[}4:0{]}} & ARTDATA\_PARITY{[}7:0{]}  & 8 x ARTDATAx{[}5:0{]}\tabularnewline
\hline 
\end{tabular}
\begin{itemize}
\item HIT\_LIST{[}31:0{]} = 32-bit list of flags corresponding to each of
the 32 VMMs. 0 - no hit, 1 - hit. A register controls if this is a
filtered (i.e. 8 hits max) or an un- filtered copy of the VMM flags
registered in a particular BC. 
\item HIT\_CNT{[}3:0{]} = number of hits (range 0 - 8; 9 - 15 invalid) 
\item ARTDATA\_PARITY{[}7:0{]} = parity bit of the ART data computed by
each of the 32 ART de-serializer units. Each bit corresponds to one
of the ART data field selected by the priority unit. 
\end{itemize}
\includegraphics[bb=0 0 200 100,draft,type=eps,bb = 0 0 200 100, draft, type=eps]{/home/nate/Desktop/Screenshot-Format of ADDC output - caproniarts@gmail.com - Gmail - Google Chrome.png}


\section{Trigger Processor Implementation and testing}

A 1/16\textsuperscript{th} slice of the complete Trigger Processor
design has been implemented and is being tested using a Xilinx VC707
Development board. This board includes a Virtex XC7VX485T-2FFG1761C
FPGA. The implementation includes 2 ADDC GBT interfaces and assoiated
trigger processor algorithim. Extrapolating from this implementation,
the resources are estimated to be \textasciitilde{}70\% of the Xilinx
485 chip. There are pin compatible upgrades to the target chip if
more resources are needed.


\subsection{Trigger Processor Initial Testing}

A 1/16\textsuperscript{th} slice of the complete Trigger Processor
design has been implemented and is being tested using a Xilinx VC707
Development board. This board includes a Virtex XC7VX485T-2FFG1761C
FPGA. The implementation includes 2 ADDC GBT interfaces and associated
trigger processor algorithm. 

To exercise the trigger processor design we have developed an ADDC
emulator. This design can be used to supply properly formatted ADDC
GBT packets through an optical transmitter as sent from an actual
ADDC. The same ART data used for simulations is being used for hardware
testing. We are also generating pseudo random ART data to test the
GBT communication and timing of packet decoder.

To evaluate the hardware implementation, we compare the hardware results
with results generated using a computer simulation. All of the differences
at this point are due to the number of significant bits being used
in the hardware implementation. We are currently working on optimizing
this. Figure show an example of such comparisons by looking at the
differences between hardware implementation and software simulation
ROI result. in this example, the x and y axis represent the percent
error of the ROI Cartesian coordinates.

We have begun integration testing with the BNL ADDC and have successfully
transfered data to the trigger processor using the ADDC's VTRx ASIC.
\end{document}
