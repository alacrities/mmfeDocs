The current latency budget for the Phase\,1 trigger processor for the
New Small Wheel, and the forward muon system in general is extremely
tight.
While the NSW trigger as designed now for installation in Long Shutdown\,2 (LS2)
already meets the Phase\,2 requirement of an angular resolution of 1\,mrad,
it is quite possible that one may further lower the thresholds
for muon momentum by taking advantage of the increased Level-1 latency time
to do a more refined calculation of muon pointing and momentum, or to
improve robustness and redundancy, e.g.\ in case of missing layers.

Currently, prompt signals from the Micromegas detectors and sTGC are
used to form track segments in the NSW. A cut is made on the pointing
back to the interaction region in a set of trigger processors, and the
resulting $\Delta\theta$, and RoI information is transmitted to the Sector Logic
which looks for a coincidence with prompt signals from the Big Wheel.
Currently, the latency budget for the NSW trigger processor algorithm
is approximately 100\,nsec, when fiber optic delays, serialization,
deserialization and other factors are taken into consideration, along
with the processing and transmission time associated with the Sector
Logic.

With a much larger amount of latency available in Phase\,2, and
anticipated advances in FPGAs, it makes sense to seriously consider a
more powerful trigger processing scheme that can take advantage of
additional processing time to make a much more refined trigger
algorithm. In this case, the Phase\,1 trigger processor hardware (ATCA cards + mezzanines),
excluding the ATCA crates and optical fibers, would be replaced.
Since such proposal is for equipment in USA\,15, it has
little to no impact on the current plans, but can provide for the
possibility of a much more refined Level-1 trigger that should be capable
of pushing down the momentum threshold for forward muons significantly
by including more fine-grained information available, given the
latency.