In addition to implementing the trigger algorithms for the \MM and
\stgc systems, the NSW Trigger Processor has to perform several
ancillary functions. These include time synchronization,
configuration, monitoring and debugging mode, among others. Several functions are common to both \MM and \stgc and their firmware can be shared as well-defined packages. These functions are:

\begin{description}

  \item[Interface for algorithm configuration parameters] \hfill \\
  Parameters for the algorithms must be stored at runtime. Examples
  are the $\Delta\theta$ and other cuts, the BCID offset, alignment
  parameters, the parts of the detector to be considered as disabled,
  road size, etc. Configuration can be done via Ethernet and the
  carrier board or via an E-link from FELIX that is available on the link that brings the TTC information.
  Readback of the parameters must also be provided.

  \item[TTC interface] \hfill \\
  FELIX provides the following TTC information on an E-link: \\
  \hspace*{5mm} Level-1 Accept, BCR, ECR, system[3..0], user[7] from the 8-bit TTC broadcast packet \\
  The E-link provides the 40\,MHz BC clock which is used to
  synchronize the output to the Sector Logic. This also includes the logic to synchronize a local bunch-crossing ID to the Sector Logic
  by means of a configurable BC offset loaded on BCR into a local BCID register.
  Note that the various input links will not have the same phase and may not even be matched to the same BC clock.
  Input processors must ensure that all sources are aligned to the same BC clock.
  Both technologies include a BCID or its low bits in the packet sent on every bunch-crossing.
  Should the offset of this BCID from the local BCID differ from what is expected, an exception message (see below) must be sent.
  The firmware to do this alignment is not shared.

  \item[Level-1 output buffer] \hfill \\
  For bunch-crossings in which at least one segment is found,
  the input data and the output segment data that is sent to the Sector Logic is stored,
  along with its BCID for later matching to the BCID of a Level-1 Accept.
  Those bunch-crossings that have Level-1 Accepts (and possibly those preceding and following) are transferred to the Level-1 output buffer (aka derandomizer).
  The data must be stored for the duration of the Level-1 latency.
  The output bandwidth should be sufficient for the rather small fixed input and output data lengths at the full Level-1 rate of 400\,kHz.
  If not, this logic could provide a BUSY output to the RODBUSY system when its output buffer becomes close to full.

  \item[Monitored event buffer] \hfill \\
  A random sample of complete events are collected for sending to a monitoring process.
  Example criteria are: any event, event with at least one segment found, events with segments outside the $\Delta\theta$ cut, \ldots\.)
  The data buffered as one event includes all the input data and the output segment data that is sent to the Sector Logic for a given BC.

  \item[Statistics buffer] \hfill \\
  Statistics are continuously collected and periodically transferred to the Statistics buffer.
  Statistics includes the number of bunch-crossings that have candidates that are not accepted by Level-1, their distribution in $R\phi$,
  the multiplicity of segments per bunch-crossing, etc.

  \item[Exception buffer] \hfill \\
  In the course of processing, exceptional conditions may be found, usually due to corrupted data.
  A convenient way to handle these is to store an exception code and some context data into a buffer which will be passed to the monitoring PC via FELIX.

  \item[Playback mode to fake input links] \hfill \\
  For development and testing we require that simulated data can be injected in place of the data received by the links to the Front End.
  One way to do this is via an E-link from FELIX that is available on the link that brings the TTC information.
  Full-speed testing.

  \item[Segment output to Sector Logic and to the ``other'' detector] \hfill \\
  Segments are sent out either to the Sector Logic via the FPGA serializer or to the ``other'' detector via a parallel LVDS bus.
  The candidate packet to be sent to the Sector Logic must be prepared from the segments found.
  Clones must be made and the output links to the Sector logic must be driven.
  If the segments found are to be sent to the ``other'' detector's Trigger Processor,
  the segment data must be sequenced out onto the parallel LVDS bus.

  \item[Merge buffers into the output GBT link to FELIX] \hfill \\
  The Level-1, Monitoring, Statistics and Exception buffers are merged, using different FELIX ``stream-IDs'' onto a fiber link to FELIX.
  FELIX then routes them to the ROD and Monitoring PCs.
  % This may use the FELIX, so-called, ``flat'' mode (not one of the GBTx modes), where data is streamed in 8b/10b encoding on a single

\end{description}

Since the MM uses the GBTx to transmit the data from the Front End and the sTGC uses native FPGA serializers,
the link interface firmware cannot be shared.
Note that the monitoring of board temperatures and voltages is done by the ATCA Shelf Manager using IPMI.
The fiber plant is described in Appendix~\ref{app:specs-fibers}.
